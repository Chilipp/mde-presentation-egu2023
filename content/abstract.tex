% !TeX root = ../mde-presentation.tex

\section{Abstract} \label{sec:abstract}

\begin{frame}[t, allowframebreaks]{Abstract}
	\begin{block}{ESM Data Exploration with the Model Data Explorer}
		\begin{small}
			Making Earth-System-Model (ESM) Data accessible is challenging due
			to the large amount of data that we are facing in this realm. The
			upload is time-consuming, expensive, technically complex, and every
			institution has their own procedures.

            Non-ESM experts face a lot of problems and pure data portals are
			hardly usable for inter- and trans-disciplinary communication of
			ESM data and findings, as this level of accessibility often
			requires specialized web or computing services.

            With the Model Data Explorer, we want to simplify the generation of
			web services from ESM data, and we provide a framework that allows
			us to make the raw model data accessible to non-ESM experts.

            Our decentralized framework implements the possibility for an
			efficient remote processing of distributed ESM data. Users
			interface with an intuitive map-based front-end to compute spatial
			or temporal aggregations, or select regions to download the data.
			The data generators (i.e. the scientist with access to the raw
			data) use a light-weight and secure python library based on the
			Data Analytics Software Framework (DASF,
			\url{https://digital-earth.pages.geomar.de/dasf/dasf-messaging-python})
			to create a back-end module. This back-end module runs close to the
			data, e.g. on the HPC-resource where the data is stored. Upon
			request, the module generates and provides the required data for
			the users in the web front-end.

		\end{small}
	\end{block}

	\framebreak

	\begin{block}{ESM Data Exploration with the Model Data Explorer}
		\begin{small}
            Our approach is intended for scientists and scientific usage! We
			aim for a framework where web-based communication of model-driven
			data science can be maintained by the scientific community. The
			Model Data Explorer ensures fair reward for the scientific work and
			adherence to the FAIR principles without too much overhead and loss
			in scientific accuracy.

            The Model Data Explorer is in the progress of development at the
			Helmholtz-Zentrum Hereon, together with multiple scientific and
			data management partners in other German research centers. The full
			list of contributors is constantly updated and can be accessed at
			\url{https://model-data-explorer.readthedocs.io}.

		\end{small}
	\end{block}
\end{frame}

